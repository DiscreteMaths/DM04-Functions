

\begin{center}
  \textbf{Tutorial Sheet 5}
\end{center}
Sets:\begin{multicols}{2}\small
  $\quad$\\
  $\mb{R}$ - All real numbers positive and negative\\
    $\mb{R}^+$ - All positive real numbers including $0$\\
    $\mb{R}^-$ - All negative real numbers including $0$
    \columnbreak
    \\$[a,b]$ - All real numbers $x$ such that $a \le x \le b$\\
    $(a,b)$ - All real numbers $x$ such that $a < x < b$\\
    $[a,\infty)$ - All real numbers $x$ such that $a \le x$\\
    $(a,\infty)$ - All real numbers $x$ such that $a < x$
  \end{multicols}
\begin{enumerate}
\item Which of the following functions are well defined functions? If the function is not well defined, give a counterexample showing that it is not.
  \begin{multicols}{2}
    \begin{enumerate}
    \item $f: \mb{R} \to \mb{R},\quad f(x)=x^2+1$
    \item $f: \mb{R}^+ \to \mb{R}^+,\quad f(x)=x^2+1$
    \item $f: \mb{R}^+ \to [1,10],\quad f(x)=x^2+1$
    \item $f: \mb{R}^+ \to [1,\infty),\quad f(x)=x^2+1$
    \item $f: \mb{R}^+ \to \mb{R}^+,\quad f(x)=\sqrt[+]{x}$
    \item $f: \mb{R}^- \to \mb{R}^-,\quad f(x)=\sqrt[+]{x}$
    \item $f: \mb{R}^+ \to \mb{R}^-,\quad f(x)=\sqrt[+]{x}$
    \item $f: \mb{R}^+ \to \mb{R},\quad f(x)=\sqrt[+]{x}$
    \item $f: \mb{R} \to \mb{R},\quad f(x)=\frac{1}{x}$
    \item $f: \mb{R}\setminus\{0\} \to \mb{R},\quad f(x)=\frac{1}{x}$
    \item $f: \mb{R}^+\setminus\{0\} \to \mb{R},\quad f(x)=\frac{1}{x}$
    \item $f: \mb{R}^+\setminus\{0\} \to \mb{R},\quad f(x)=\frac{1}{x-1}$
    \item $f: \mb{R}^+\setminus\{1\} \to \mb{R}^+,\quad f(x)=\frac{1}{x-1}$
    \item $f: \mb{R} \to \mb{R},\quad f(x)=e^x$
    \item $f: \mb{R} \to \mb{R}^+,\quad f(x)=e^x-1$
    \item $f: \mb{R} \to \mb{R},\quad f(x)=\ln(x)$
    \item $f: \mb{R}^+\setminus \{0\} \to \mb{R}^+,\quad f(x)=\ln(x)$
    \item $f: \mb{R}^+\setminus \{0\} \to \mb{R},\quad f(x)=\ln(x)$
    \item $f: (1,\infty) \to \mb{R},\quad f(x)=\ln(x+1)$
    \end{enumerate}
  \end{multicols}

\item
  For each of the following well defined functions, say whether the function is one-to-one, onto, or invertible. In the case of invertible functions, give the inverse function. In the case of non-invertible functions, modify the domain and codomain of the functions to make them invertible and give the corressponding inverse function.
  \begin{multicols}{2}
    \begin{enumerate}
    \item $f: \mb{R} \to \mb{R},\quad f(x)=2x+4$
    \item $f: \mb{R} \to \mb{R},\quad f(x)=x$
    \item $f: \mb{R} \to \mb{R},\quad f(x)=x^2$
    \item $f: \mb{R} \to \mb{R}^+,\quad f(x)=x^2+4$
    \item $f: \mb{R}^+ \to \mb{R},\quad f(x)=\sqrt[+]{x}$
    \item $f: \mb{R}\setminus\{0\} \to \mb{R},\quad f(x)=\frac{1}{x}$
    \item $f: \mb{R} \to \mb{R},\quad f(x)=e^x$
    \item $f: \mb{R}^+ \to [1,\infty),\quad f(x)=e^x$
    \item $f: \mb{R}^+ \to \mb{R}^+,\quad f(x)=e^x+1$
    \item $f: \mb{R} \to \mb{R},\quad f(x)=\sin(x)$
    \item $f: (\text{-}\pi,\pi) \to [\text{-}1,1],\  f(x)=\sin(x)$
    \item $f: (\text{-}\frac{\pi}{2},\frac{\pi}{2}) \to [\text{-}1,1],\  f(x)=\sin(x)$
    \end{enumerate}
  \end{multicols}

\item
  Graph the well defined function $f: \mb{R} \to \mb{R}, \ f(x)=\cosh(x)$ on the interval $[-2,2]$. Based on the graph, give a suitable domain and codomain of the function to make it invertible.

\pagebreak

\item For each of the following graphs,
  \begin{enumerate}
    \item  Use the vertical line test to determine whether it is a graph of a well defined function mapping subsets of the reals to the reals.
    \item  Use the horizontal line test to determine over which domains and codomains(on the graph) the function is one-to-one, onto, or invertible.
  \end{enumerate}


  \begin{figure}[h!]
    \centering
    \subfloat[$y=ax+b$]{\includegraphics[width=0.45\textwidth]{functut_vht_1.eps}} \hspace{1em}
    \subfloat[$y=ax^2$]{\includegraphics[width=0.45\textwidth]{functut_vht_2.eps}}\\
    \subfloat[$y^2=x$]{\includegraphics[width=0.45\textwidth]{functut_vht_3.eps}} \hspace{1em}
    \subfloat[$\frac{x^2}{a^2}+\frac{y^2}{b^2}=1$]{\includegraphics[width=0.45\textwidth]{functut_vht_4.eps}}\\
    \subfloat[$y=\sin(x)$]{\includegraphics[width=0.45\textwidth]{functut_vht_5.eps}} \hspace{1em}
    \subfloat[$y=e^x$]{\includegraphics[width=0.45\textwidth]{functut_vht_6.eps}}\\
    \subfloat[$y=\cosh(x)$]{\includegraphics[width=0.4\textwidth]{functut_vht_7.eps}} \hspace{1em}
    \subfloat[$y=\sinh(x)$]{\includegraphics[width=0.4\textwidth]{functut_vht_8.eps}}\\
  \end{figure}

