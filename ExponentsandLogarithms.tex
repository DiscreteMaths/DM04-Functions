\documentclass{beamer}

\usepackage{amsmath}
\usepackage{amssymb}

\begin{document}
\begin{frame}
\frametitle{Revision Topics}
\begin{itemize}
\item Exponents
\item Laws of Logarithms
\item The Natural Logarithm
\end{itemize}
\end{frame}
%========================================================================= %
\section{Laws of Logarithms}
\begin{frame}
\frametitle{LAws of Logarithms}
% %\subsection*{Trigonometric Subsititution}
\[\int\frac{dx}{\sqrt{a^2-x^2}}\]

\[x=a\sin(\theta),\quad dx=a\cos(\theta)\,d\theta, \quad \theta=\arcsin\left(\frac{x}{a}\right)\]

{\large
	\begin{eqnarray}
	\int\frac{dx}{\sqrt{a^2-x^2}} & = \int\frac{a\cos(\theta)\,d\theta}{\sqrt{a^2-a^2\sin^2(\theta)}}\\ \nonumber &= \int\frac{a\cos(\theta)\,d\theta}{\sqrt{a^2(1-\sin^2(\theta))}} \\ \nonumber
	& = \int\frac{a\cos(\theta)\,d\theta}{\sqrt{a^2\cos^2(\theta)}} \\ &= \int d\theta=\theta+C \\ \nonumber &=\arcsin\left(\frac{x}{a}\right)+C
	\end{eqnarray}
}
\end{frame}



%------------------------------------------------------------------------- %
% Laws of Logarithms


\begin{frame}
\frametitle{Logarithms :  Change of Base}
\Large

\end{frame}
%============================================================================================================================ %

\section{Partial Derivatives: Volume of a Cone}
\begin{frame}
\frametitle{Partial Derivatives: Volume of a Cone}
\Large
The volume ''V'' of a cone depends on the cone's height ''h'' and its radius 'r' according to the formula
\[V(r, h) = \frac{\pi r^2 h}{3}.\]
The partial derivative of ''V'' with respect to 'r' is
\[\frac{ \partial V}{\partial r} = \frac{ 2 \pi r h}{3},\]

which represents the rate with which a cone's volume changes if its radius is varied and its height is kept constant.

\end{frame}



%------------------------------------------------------------------------- %
\begin{frame}
	\frametitle{Partial Derivatives: Volume of a Cone}
	\Large
The partial derivative with respect to ''h'' is
\[\frac{ \partial V}{\partial h} = \frac{\pi r^2}{3},\]

which represents the rate with which the volume changes if its height is varied and its radius is kept constant.
\end{frame}

%================================================================================= %

\section*{Numerical Methods}

\begin{frame}
	\frametitle{Trapezoidal Rule}
\[\int_{a}^{b} f(x)\, dx \approx (b-a)\frac{f(a) + f(b)}{2}\]

\end{frame}
%================================================================================= %
\section{Calculus}
\begin{frame}
\frametitle{Newton-Raphson Method}
\Large
\[x_{1} = x_0 - \frac{f(x_0)}{f'(x_0)}\]
\[x_{n+1} = x_n - \frac{f(x_n)}{f'(x_n)} \]
\end{frame}

%------------------------------------%
\begin{frame}
\frametitle{Partial Fraction Expansion}
\begin{itemize}
	\item Distinct Linear Factors
	\item Repeated Linear Factors
	\item Distinct Quadratic Factors
	\item Repeated Quadratic Factors
\end{itemize}
\end{frame}

%------------------------------------%
\begin{frame}
	\frametitle{{Partial Differentiation}
\[\frac{\partial f}{\partial x}\]

\[z = f(x, y) = \,\! x^2 + xy + y^2.\]
\[\frac{\partial z}{\partial x} = 2x+y\]
\[\frac{\partial z}{\partial x} = 3\]
\end{frame}
\end{document}