\documentclass{beamer}
\usepackage{amsmath}
\usepackage{amssymb}

% % - http://tutorial.math.lamar.edu/Classes/CalcI/InverseFunctions.aspx

\begin{document}
%===========================================================================%
\begin{frame}[fragile]
	\frametitle{Inverse of Functions}
	\Large
In the last example from the previous section we looked at the two functions  and  and saw that


and as noted in that section this means that there is a nice relationship between these two functions.  Let’s see just what that relationship is.  Consider the following evaluations.

\end{frame}
%===========================================================================%
\begin{frame}[fragile]
\frametitle{Inverse of Functions}
\Large

In the first case we plugged  into  and got a value of -5.  We then turned around and plugged  into  and got a value of -1, the number that we started off with. 
\end{frame}
%===========================================================================%
\begin{frame}[fragile]
	\frametitle{Inverse of Functions}
	\Large
In the second case we did something similar.  Here we plugged  into  and got a value of , we turned around and plugged this into  and got a value of 2, which is again the number that we started with.

Note that we really are doing some function composition here.  The first case is really,

and the second case is really,


\end{frame}
%===========================================================================%
\begin{frame}[fragile]
\frametitle{Inverse of Functions}
\Large
Note as well that these both agree with the formula for the compositions that we found in the previous section.  We get back out of the function evaluation the number that we originally plugged into the composition.

So, just what is going on here?  In some way we can think of these two functions as undoing what the other did to a number.  In the first case we plugged  into  and then plugged the result from this function evaluation back into  and in some way  undid what  had done to  and gave us back the original x that we started with.
\end{frame}
%===========================================================================%
\begin{frame}[fragile]
	\frametitle{Inverse of Functions}
	\Large
\begin{itemize}
\item Function pairs that exhibit this behavior are called inverse functions. 
\item Before formally defining inverse functions and the notation that we’re going to use for them we need to get a definition out of the way.   

\item A function is called \textbf{\textit{one-to-one}} if no two values of x produce the same y.  
\item Mathematically this is the same as saying,
\end{itemize}

\end{frame}
%===========================================================================%
\begin{frame}[fragile]
	\frametitle{Inverse of Functions}
	\Large
So, a function is one-to-one if whenever we plug different values into the function we get different function values.

Sometimes it is easier to understand this definition if we see a function that isn’t one-to-one.  Let’s take a look at a function that isn’t one-to-one.  The function  is not one-to-one because both  and .  In other words there are two different values of x that produce the same value of y.  Note that we can turn  into a one-to-one function if we restrict ourselves to .  This can sometimes be done with functions.
\end{frame}
%===========================================================================%
\begin{frame}[fragile]
	\frametitle{Inverse of Functions}
	\Large
Showing that a function is one-to-one is often tedious and/or difficult.  For the most part we are going to assume that the functions that we’re going to be dealing with in this course are either one-to-one or we have restricted the domain of the function to get it to be a one-to-one function.

\end{frame}
%===========================================================================%
\begin{frame}[fragile]
	\frametitle{Inverse of Functions}
	\Large
	
Now, let’s formally define just what inverse functions are.  Given two one-to-one functions  and  if


then we say that  and  are inverses of each other.  More specifically we will say that  is the inverse of  and denote it by

\end{frame}
%===========================================================================%
\begin{frame}[fragile]
	\frametitle{Inverse of Functions}
	\Large
Likewise we could also say that  is the inverse of  and denote it by



The notation that we use really depends upon the problem.  In most cases either is acceptable.
\end{frame}
%===========================================================================%
\begin{frame}[fragile]
	\frametitle{Inverse of Functions}
	\Large
For the two functions that we started off this section with we could write either of the following two sets of notation.



Now, be careful with the notation for inverses.  The “-1” is NOT an exponent despite the fact that is sure does look like one!  When dealing with inverse functions we’ve got to remember that
\end{frame}
%===========================================================================%
\begin{frame}[fragile]
	\frametitle{Inverse of Functions}
	\Large

This is one of the more common mistakes that students make when first studying inverse functions.

The process for finding the inverse of a function is a fairly simple one although there are a couple of steps that can on occasion be somewhat messy.  Here is the process
\end{frame}
%===========================================================================%
\begin{frame}[fragile]
	\frametitle{Inverse of Functions}
	\Large
Finding the Inverse of a Function
Given the function  we want to find the inverse function, .
First, replace  with y.  This is done to make the rest of the process easier.
Replace every x with a y and replace every y with an x.
Solve the equation from Step 2 for y.  This is the step where mistakes are most often made so be careful with this step.
Replace y with .  In other words, we’ve managed to find the inverse at this point!

\end{frame}
%===========================================================================%
\begin{frame}[fragile]
	\frametitle{Inverse of Functions}
	\Large
Verify your work by checking that  and  are both true.  This work can sometimes be messy making it easy to make mistakes so again be careful.

That’s the process.  Most of the steps are not all that bad but as mentioned in the process there are a couple of steps that we really need to be careful with since it is easy to make mistakes in those steps.
\end{frame}
%===========================================================================%
\begin{frame}[fragile]
	\frametitle{Inverse of Functions}
	\Large
In the verification step we technically really do need to check that both  and  are true.  For all the functions that we are going to be looking at in this course if one is true then the other will also be true.  However, there are functions (they are beyond the scope of this course however) for which it is possible for only one of these to be true.  This is brought up because in all the problems here we will be just checking one of them.  We just need to always remember that technically we should check both.
\end{frame}
%===========================================================================%
\begin{frame}[fragile]
	\frametitle{Inverse of Functions}
	\Large
Let’s work some examples.

Example 1  Given   find .

Solution
Now, we already know what the inverse to this function is as we’ve already done some work with it.  However, it would be nice to actually start with this since we know what we should get.  This will work as a nice verification of the process.

So, let’s get started.  We’ll first replace  with y.

\end{frame}
%===========================================================================%
\begin{frame}[fragile]
	\frametitle{Inverse of Functions}
	\Large
Next, replace all x’s with y and all y’s with x.


Now, solve for y.

\end{frame}
%===========================================================================%
\begin{frame}[fragile]
	\frametitle{Inverse of Functions}
	\Large
Finally replace y with .


\end{frame}
%===========================================================================%
\begin{frame}[fragile]
	\frametitle{Inverse of Functions}
	\Large
Now, we need to verify the results.  We already took care of this in the previous section, however, we really should follow the process so we’ll do that here.  It doesn’t matter which of the two that we check we just need to check one of them.  This time we’ll check that  is true.

\end{frame}
%===========================================================================%
\begin{frame}[fragile]
	\frametitle{Inverse of Functions}
	\Large
Example 2  Given   find .

Solution
The fact that we’re using  instead of  doesn’t change how the process works.  Here are the first few steps.


Now, to solve for y we will need to first square both sides and then proceed as normal.

\end{frame}
%===========================================================================%
\begin{frame}[fragile]
	\frametitle{Inverse of Functions}
	\Large
This inverse is then,


Finally let’s verify and this time we’ll use the other one just so we can say that we’ve gotten both down somewhere in an example.



So, we did the work correctly and we do indeed have the inverse.

The next example can be a little messy so be careful with the work here.
\end{frame}
%===========================================================================%
\begin{frame}[fragile]
	\frametitle{Inverse of Functions}
	\Large
Example 3  Given   find .
Solution
The first couple of steps are pretty much the same as the previous examples so here they are,


Now, be careful with the solution step.  With this kind of problem it is very easy to make a mistake here.


So, if we’ve done all of our work correctly the inverse should be,

\end{frame}
%===========================================================================%
\begin{frame}[fragile]
	\frametitle{Inverse of Functions}
	\Large
Finally we’ll need to do the verification.  This is also a fairly messy process and it doesn’t really matter which one we work with.


Okay, this is a mess.  Let’s simplify things up a little bit by multiplying the numerator and denominator by .

\end{frame}
%===========================================================================%
\begin{frame}[fragile]
	\frametitle{Inverse of Functions}
	\Large
Wow.  That was a lot of work, but it all worked out in the end.  We did all of our work correctly and we do in fact have the inverse.

There is one final topic that we need to address quickly before we leave this section.  There is an interesting relationship between the graph of a function and the graph of its inverse.
\end{frame}
%===========================================================================%
\begin{frame}[fragile]
	\frametitle{Inverse of Functions}
	\Large
Here is the graph of the function and inverse from the first two examples. 

% % InverseFcn_G3

In both cases we can see that the graph of the inverse is a reflection of the actual function about the line .  This will always be the case with the graphs of a function and its inverse.


\end{frame}
%===========================================================================%
\end{document}

